\section{Introduction}

Recognition of handwritten text has been a popular research area for decades because it can be used in so many different applications.
Formally, handwritten recognition is the task of transforming a language represented in graphical form into its symbolic representation \cite{introsurvey}.
There are two different approaches to handwritten recognition, \textit{online} and \textit{offline}.
In the online approach we know the order in which the strokes and individual points were drawn.
This information can easily be captured if the text is recorded by a digital pen or on a touchscreen.
In the offline approach we are only given the final image.
Online recognition is primarily used for signature verification, author authentication and digital pens.
Application areas for offline recognition include postal automation, bank cheque processing and automatic data entry \cite{intro1}.
In this paper we only consider offline handwritten recognition.

The ultimate goal in handwritten recognition is to recognize words.
However, one way to potentially decompose or simplify the problem is to segment words into its individual characters \cite{intro-Yacoubi}. 
Segmentation can either be done \textit{explicitly} or \textit{implicitly}.
Explicit segmentation tries to separate the word at character boundaries while implicit segmentation separates the word into equal sized frames.
The implicit frames, each represented by a feature vector, are then mapped into characters.

\section{Previous work}
Because handwritten recognition is such a well-researched area there is a wealth of literature available.
We mention only a few references that we found helpful.
Cheriet et al. \cite{Cheriet} gives a good review of the development of handwritten recognition.
They also go on to give a broad overview of feature extraction and classification using a plethora of different techniques.
Rabiner, L. R. \cite{Rabiner1989} gives an excellent review of Hidden Markov Models (HMM) and the Baum-Welch training algorithm, as well as how to apply them in speech recognition.
El-Yacoubi et al. \cite{intro-Yacoubi} introduce an approach to recognize text using Hidden Markov Models with explicit word segmentation.
Laan et al. \cite{initialmodel} covers different initial model selection techniques for the Baum-Welch algorithm.
Despite impressive progress over the last couple of decades, performance is still far away from human performance.


