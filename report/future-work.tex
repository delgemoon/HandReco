%. Top down feature extraction to improve performance.
%. Better scaling that doesn't thicken lines, or second step to remove unnecessary pixels.
%. More advanced feature extraction, use different representation that makes it easier to extend with more features.

%. Add more training data.
%. Consider non pre-segmented.

While the results show that using HMMs for building a handwritten recognition system is a viable alternative, there is a lot of room for improvement.
One way to potentially improve performance is by extending the feature extraction to consider segments from top to bottom as well as from left to right.
This means that the observation sequence would become twice as long, given a square image.

During scaling, the lines can become wider than one pixel.
Because we are categorizing the strokes based on how many pixels they contain, this is a problem.
This could be fixed by adding a thinning phase after the scaling is completed to make the strokes one pixel wide again.

Another way to improve performance is by using a different feature representation.
For example, vector quantization can be used to map vectors into a smaller space which can then be used as observations in the HMM.
Using this method we would not be reliant on arbitrary constants to map features into observations.
It would also make it easier to the extend the system to use additional features.

We saw in our results that adding more training data improves performance so this is an easy way to enhance the system.
Finally, to make the system more general we should allow for words that are not pre-segmented into characters.