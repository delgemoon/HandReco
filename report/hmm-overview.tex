
For pattern recognition problem, like handwritten image recognition, there will  always be some randomness and uncertainty from the source recognition data. Stocastic modeling deals with these problem efficiently by using probabilistic models \cite{Cho1995}.  Among such stochastic approaches, Hidden Markov Models have been widely used to model dynamic signals.
The Hidden Markov Model treats the data as a sequence of observations, while using hidden states that are connected to each other by transition probabilities.
 
An HMM is characterized by the following \cite{Rabiner1989}:
\begin{enumerate}
\item	N, the number of states in the model.
\item	M, the number of distinct observation symbols. % per state. I'm pretty sure this is not per state, but global?
\item	A, the transition probability distribution.
\item	B, the observation symbol probability distribution for each state
\item	$\pi$, the initial state state distribution.
\end{enumerate}

In contrast to a knowledge-based approach, HMMs use statistical algorithms that can automatically extract knowledge from samples. 
%% I think the following sentence is a little unclear what is meant
Also, HMMs model patterns implicitly with different paths in the stochastic work. 
The modeling power can be enhanced by adding more samples \cite{Cho1995}.
